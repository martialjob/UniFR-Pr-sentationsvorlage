\usepackage{ 
	amsmath,	% Standard Mathe-Paket
        amssymb,    % Noch ein Paar mathematische Symbole.
	amsfonts,	% Ergänzt mathematische Symbole.
	paralist,	% Erweiterung der bereits bestehenden Listenumgebungen
	listings,	% Code-Formatierung 
	setspace,	% Zeilenabstand
	courier,	% Monospace Schrift
	mathptmx,	% Times New Roman für Lauftext und Überschriften
	titlesec,   % Um Titel zu formatieren.
	lipsum,     % Lorem Ipsum
	subfiles,   % Um subfiles ins Hauptdokument zu integrieren.
	url,        % Dient zur Auszeichnung und Erstellung von Hyperlinks
	tabularx,   % Ermöglicht Seitenbreite (linewidth/textwidth) Tbl.
        colortbl,   % Tabellen färben <3
        fontspec,
        array       % Lässt neue Spalten definieren
}

\usepackage[utf8]{inputenc}  % Eingabekodierung
\usepackage[ngerman]{babel}  % Sprache (Deutsch, Neue Rechtschreibung)
\usepackage[none]{hyphenat}  % KEINE Wortumbrüche
\usepackage[T1]{fontenc}    % Umlaute, Akzente,...
\usepackage[
    autostyle=true,
    german=swiss]
    {csquotes} % Anführungszeichen mit \enquote.
\usepackage[
    backend=biber,
    style=apa,
    sorting=nyt]
    {biblatex} % Biblatex Settings (APA-Deutsch)

\usefonttheme{serif}
\setmainfont{Arial}

% Definiert die linke und rechte margins.
\setbeamersize{text margin left=5mm, text margin right=5mm} 

% Definiert die Footline.
\setbeamertemplate{footline}[text line]{%
  \parbox{\linewidth}{\vspace*{-16pt}\hspace{0,4\linewidth}\insertshortinstitute\ \mid \insertshorttitle \mid \insertshortdate \hfill\insertpagenumber}}
\setbeamertemplate{navigation symbols}{}



\definecolor{highlight1}{HTML}{344A9A} %dunkelblau
\definecolor{highlight2}{HTML}{A3B1E0} %hellblau
\definecolor{highlight3}{HTML}{8F6B30} %braun
\definecolor{highlight4}{HTML}{FFE862} %gelb
\definecolor{highlight5}{HTML}{F5C2ED} %rosa
\definecolor{highlight6}{HTML}{00997D} %grün

\setbeamercolor{title}{fg=white}
\setbeamercolor{author}{fg=white}
%\setbeamercolor{institute}{fg=white}
\setbeamercolor{frametitle}{fg=white}
\setbeamercolor{date}{fg=white}
\setbeamercolor{footline}{fg=highlight1}

% Biblatex Referenzen werden damit formatiert.
\DeclareLanguageMapping{ngerman}{ngerman-apa}


% Fügt im Literaturverzeichnis "Verfügbar unter" und optional "Letzter Zugriff" 
% zu .bib entries mit url und urldate hinzu.
\DeclareFieldFormat{formaturl}{Verfügbar unter #1}
\DeclareFieldFormat{formatdate}{Letzter Zugriff: #1}

%\newbibmacro*{url+urldate}{%
%	\iffieldundef{urlyear}{%
%		\printtext[formaturl]{\printfield{url}}\nopunct%
%	}{%
%		\printtext[formaturl]{\printfield{url}}\adddot\space%
%		\printtext[formatdate]{\printurldate}%
%	}%
%}

% Inline citations in format Author (year), S. 11
\newcommand\mycite[2][]{\citeauthor{#2}\ (\citeyear{#2})\ifx#1\undefined\else, #1\fi}


% Ergänzt im Literaturverzeichnis den Publisher mit Ort (Ort, Herausgeber)
\DeclareListFormat{publisher}{\printlist{location}\addcomma\space #1}

% Um etwas in Anführungszeichen zu setzen, ohne, dass babel Umlaute erzeugt.
\newcommand{\myquote}[1]{\glqq{}#1\grqq{}}

%\newcommand*{\sectionpage}{\usebeamertemplate*{section page}}

\AtBeginSection[]{
    \begin{frame}<beamer>
        \frametitle{currentsection}
        \usebackgroundtemplate{\includegraphics[width=\paperwidth]{settings/templates/bg_chapter.pdf}}
    \end{frame}
}